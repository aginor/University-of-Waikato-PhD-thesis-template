% to allow unicode input, eg in bibliography
\usepackage[utf8]{inputenc}

\usepackage{ifthen}

% more compact enumerate and itemize
\usepackage{mdwlist}

% to allow inclusion of version information from svn
\usepackage{svn-multi}

\usepackage{url}
\usepackage{graphicx}

% ensure floats come after their first reference
\usepackage{flafter}

% prettier tables
\usepackage{booktabs}

% place footnotes *below* floats at bottom of page
\usepackage{stfloats}
\fnbelowfloat

% new type of float for examples
\usepackage{float}
\floatstyle{komabelow}
\newfloat{xmpl}{tbp}{loe}[chapter]
\floatname{xmpl}{Example}
\newcommand{\theHxmpl}{\arabic{xmpl}}

% from http://wiki.lyx.org/FAQ/ShadedBox
% except that we're using xcolor cause that helps with zebra-striped
% tables, see
% http://en.wikibooks.org/wiki/LaTeX/Tables#Alternate_Row_Colors_in_Tables
\makeatletter
\usepackage[table]{xcolor}
\usepackage{calc}
\definecolor{shade}{gray}{0.9}
\definecolor{frame}{gray}{0.8}
\setlength{\fboxrule}{2pt}
\setlength{\fboxsep}{0.5em}
\newenvironment{myshadedbox}[1][]%
{%
  \normalfont\small\sffamily\raggedright
  \begin{center}
    \vspace{-2ex}
  \begin{lrbox}{\@tempboxa}%
    \begin{minipage}{0.95\textwidth-2\fboxsep-2\fboxrule}
}%
{%
  \end{minipage}%
  \end{lrbox}%
  \fcolorbox{frame}{shade}{\usebox{\@tempboxa}}
  \end{center}
  \vspace{-2ex}
}%
\makeatother

% for zebra-striped tables
\definecolor{oddrowcolor}{gray}{0.9}
\definecolor{evenrowcolor}{gray}{1}

% for code listings
\usepackage{listings}
\lstdefinelanguage{n3}{%
  morekeywords=[2]{@prefix},
  morekeywords=[3]{(,),[,],;,.},
  morekeywords=[4]{a,has,is,of},
  sensitive=false,
  morestring=[b]",
  emph={^^xsd:nonNegativeInteger,^^xsd:string},
  alsoletter={()[];.:^},
}[keywords,comments,strings,emph]
\lstset{language=n3}
\lstset{%
  basicstyle=\small\ttfamily,
  keywordstyle={\color{listingsthing}},
  keywordstyle=[2]{\color{listingsprefix}},
  keywordstyle=[3]{\color{listingsinterpunct}},
  keywordstyle=[4]{\color{listingskeyword}\bfseries},
  emphstyle={\color{listingsnamespace}},
  showstringspaces=false,
  stringstyle={\color{listingsstring}},
  breaklines=true
}
\definecolor{listingsthing}{RGB}{175,126,21}
\definecolor{listingsprefix}{gray}{0}
% parrot teal
%\definecolor{listingskeyword}{RGB}{52,112,125}
\definecolor{listingskeyword}{gray}{0}
\definecolor{listingsinterpunct}{gray}{0}
% parrot dirty yellowish
%\definecolor{listingsnamespace}{RGB}{186,203,206}
\definecolor{listingsnamespace}{gray}{0}
\definecolor{listingsstring}{RGB}{52,112,125}

\definecolor{owlcolor}{RGB}{52,112,125}

% to save space in some tables
\usepackage{rotating}
\newcommand*{\rotatedcolheader}[1]{\begin{sideways}#1\end{sideways}}

% for acronyms
\usepackage{acronym}[2009/10/20]
\acrodef{carpe}[CARPE]{Continuous Archival and Retrieval of Personal
  Experiences}
\acrodef{pct}[PCT]{Personal Chronicling Tools}
\acrodef{pim}[PIM]{Personal Information Management}

% to separate things in the online version from things in the print
% version -- dirty hack, swap definitions for print version
\newcommand{\onlinevsprint}[2]{#2}
%\newcommand{\onlinevsprint}[2]{#1} 

% Table of contents only goes down to subsection
\setcounter{tocdepth}{2}

% accommodate 3-digit page numbers
% see http://www.mofeel.net/809-comp-text-tex/11299.aspx
% and http://www.komascript.de/node/608
\makeatletter% --> De-TeX-FAQ
\renewcommand*{\@tocrmarg}{3em}
\renewcommand*{\@pnumwidth}{2em}
\makeatother% --> \makeatletter

% improve spacing (?)
\makeatletter
\renewcommand\section{\@startsection{section}{1}{\z@}%
  {-2.5ex \@plus -.5ex \@minus -.2ex}%
  {1.8ex \@plus.2ex}%
  {\setlength{\parfillskip}{\z@ \@plus 1fil}%
    \raggedsection\normalfont\sectfont\nobreak\size@section\nobreak}}
\renewcommand\subsection{\@startsection{subsection}{2}{\z@}%
  {-.8ex\@plus -.5ex \@minus -.2ex}%
  {1.3ex \@plus .2ex}%
  {\setlength{\parfillskip}{\z@ \@plus 1fil}%
    \raggedsection\normalfont\sectfont\nobreak\size@subsection\nobreak}}
\renewcommand\subsubsection{\@startsection{subsubsection}{3}{\z@}%
  {-.5ex\@plus -.3ex \@minus -0ex}%
  {.8ex \@plus .2ex}%
  {\setlength{\parfillskip}{\z@ \@plus 1fil}%
    \raggedsection\normalfont\sectfont\nobreak\size@subsubsection\nobreak}}
\renewcommand\paragraph{\@startsection{paragraph}{4}{\z@}%
  {-0ex \@plus -.5ex \@minus -0ex}%
  {-1em}%
  {\setlength{\parfillskip}{\z@ \@plus 1fil}%
    \raggedsection\normalfont\sectfont\size@paragraph}}
\makeatother

% change size of floats allowed at top/bottom etc
% http://www.tex.ac.uk/cgi-bin/texfaq2html?label=floats
\renewcommand{\topfraction}{.85}
\renewcommand{\bottomfraction}{.7}
\renewcommand{\textfraction}{.15}
\renewcommand{\floatpagefraction}{.66}
\renewcommand{\dbltopfraction}{.66}
\renewcommand{\dblfloatpagefraction}{.66}
\setcounter{topnumber}{9}
\setcounter{bottomnumber}{9}
\setcounter{totalnumber}{20}
\setcounter{dbltopnumber}{9}

% hyperref wants to be added as the very last usepackage
\usepackage{ifpdf}
\ifpdf
  \usepackage[pdftex,%
  backref=false,%
  linktocpage=false,%
  breaklinks,%
  colorlinks,%
  linkcolor=black,%
  pagecolor=black,%
  urlcolor=black,%
  citecolor=black,%
  pdfauthor=Firstname~Lastname,%
  pdftitle=Title~of~the~thesis,%
  plainpages=false%
  ]{hyperref}
  \usepackage[final]{pdfpages}
\else
  \usepackage[dvips,%
  backref=false,%
  linktocpage,%
  breaklinks,%
  colorlinks,%
  linkcolor=black,%
  pagecolor=black,
  urlcolor=black,%
  citecolor=black%
  ]{hyperref}
\fi 

% delete these two lines if you want monospace URLs
\urlstyle{sf}
\UrlFont{\sffamily\small}

% to include svn version information, see chapter files for example
\newcommand{\versionfootnote}{\footnote{rev.~\svnfilerev, last updated \svnfiledate.}}
%\newcommand{\versionfootnote}{}

% custom environments and commands -- you can delete everything below
% this line
\newcounter{recommend}
\newcommand{\recommendation}[1]{%
\begin{quotation}%
\stepcounter{recommend}
\noindent\textit{R\therecommend: }#1%
\end{quotation}}

\newcommand{\notetoself}[1]{$\bullet$~{\color{magenta}\sffamily#1}~$\bullet$}
\newcommand{\refmissing}{{\small\textbf{\color{red}(ref!)}}}
\newcommand*{\picmissing}[1]{{\color{cyan}\sffamily#1}}

\newcommand{\defi}[1]{\emph{#1}}

\newcommand*{\owl}[1]{\textsf{\small\color{owlcolor}#1}}

\newcounter{walkthroughstep}[subsection]
\newenvironment{step}{\refstepcounter{walkthroughstep}\par\medskip\noindent%
\begin{tabular*}{\textwidth}{p{.95\textwidth}}%
\toprule%
{\itshape Step~\thewalkthroughstep:~}%
}{%
\\%
\bottomrule\end{tabular*}\par\medskip%
}

\newenvironment{answer}[1]{%
  \sffamily\small%
  \setlength{\parindent}{0pt}%
  \par\hrulefill\par%
  {\bfseries #1}\par\vspace{-\parsep}%
  \hrulefill\par%
  \nopagebreak[4]
}{%
  \nopagebreak[4]
  \par\vspace{-\parsep}\hrulefill\par%
}

% custom hyphenation -- adjust as necessary
\hyphenation{tem-po-ral se-quen-ces re-cor-ded auto-bio-gra-phi-cal
  epi-so-dic screen-shot screen-shots}

%%% Local Variables: 
%%% mode: latex
%%% TeX-master: "thesis"
%%% End: 
